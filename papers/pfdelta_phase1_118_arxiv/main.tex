\documentclass[11pt,a4paper]{article}

%% ── Packages ────────────────────────────────────────────────────────────────
\usepackage[margin=2.5cm]{geometry}
\usepackage{amsmath,amssymb,amsfonts}
\usepackage{booktabs}
\usepackage{graphicx}
\usepackage[T1]{fontenc}
\usepackage[utf8]{inputenc}
\usepackage{microtype}
\usepackage{parskip}
\usepackage[dvipsnames,svgnames,table]{xcolor}
\usepackage{hyperref}
\hypersetup{
  colorlinks=true,
  linkcolor=MidnightBlue,
  citecolor=MidnightBlue,
  urlcolor=MidnightBlue,
  pdftitle={Evolutionary Parameter Adaptation for Streaming Power Flow Robustness},
  pdfauthor={Dumitru Novic},
  pdfkeywords={evolutionary computation, power flow, streaming optimization,
               frozen elites, Newton-Raphson, nonlinear surrogate, N-1 contingency}
}
\usepackage[numbers,sort&compress]{natbib}
\usepackage{caption}
\usepackage{subcaption}
\usepackage{algorithm}
\usepackage{algpseudocode}
\usepackage{enumitem}
\usepackage{tcolorbox}
\tcbuselibrary{skins}
\usepackage{array}
\usepackage{multirow}
\usepackage{siunitx}
\sisetup{detect-all}

%% ── Color definitions ────────────────────────────────────────────────────────
\definecolor{KeyBox}{RGB}{240,248,255}
\definecolor{KeyBorder}{RGB}{70,130,180}

%% ── Title block ─────────────────────────────────────────────────────────────
\title{\textbf{Evolutionary Parameter Adaptation for Streaming Power Flow\\
Robustness under Multi-Zone Shocks and N-1 Topology Events}\\[0.4em]
{\large Phase 1: 118-Bus Nonlinear Surrogate Evaluation}}

\author{
  Dumitru Novic\\[0.2em]
  \textit{Independent Researcher}\\[0.2em]
  \small\href{mailto:d.novic@inzori.ai}{d.novic@inzori.ai}
}

\date{February 2026}

%% ─────────────────────────────────────────────────────────────────────────────
\begin{document}

\maketitle

\begin{abstract}
We study whether an evolutionary computation framework can learn and maintain
effective Newton-Raphson (NR) solver parameters in a streaming, continuously-disturbed
power-flow (PF) setting, where classical fixed heuristics fail under tight iteration budgets.
This phase uses a controlled nonlinear surrogate inspired by PFΔ's motivation
\citep{Rivera2025pfidelta}, not a full AC power flow reproduction.
The surrogate models the 118-bus network topology with cubic nonlinearity, regional
multi-zone shocks, seasonal drift, and rare N-1 line-outage events, creating a
genuine compute-limited regime.

We introduce \textbf{inZORi}, an autonomous evolutionary framework
\citep{ZORFramework2025} in which a population of organisms continuously propose
NR initialisation parameters (damping factor, warm-start mixture) subject to
energy-based selection.  We compare inZORi against two deterministic baselines
under two fairness controls (equal PF budget; equal wall-clock time), across
30 seeds and 50\,000 streaming steps.

\textbf{Key results}: inZORi achieves 99.1\% S3 (severe) convergence versus
83.9\% for a periodic-reset baseline, a \textbf{+15.2 percentage point (pp)} improvement
(CI95: [98.9\%,\,99.3\%] vs [83.7\%,\,84.2\%], non-overlapping).
After N-1 topology events, inZORi recovers in 1.82 steps compared to 12.91 for
the periodic-reset baseline (\textbf{$\sim$7$\times$ faster}).
We further demonstrate a \textbf{Frozen-Elites} deployment mode: pre-trained elite
genomes, deployed without any online evolution, achieve \emph{100\%} S3 convergence
with K\,=\,1 PF solve per step and 443 converged steps per second—exceeding both
baselines in convergence and throughput simultaneously.
\end{abstract}

%% ── Key-Results Box ──────────────────────────────────────────────────────────
\begin{tcolorbox}[
  enhanced,
  title={\textbf{Key Results} (118-bus, severe\_plus + N-1, verified)},
  colback=KeyBox,
  colframe=KeyBorder,
  fonttitle=\bfseries\small,
  coltitle=white,
  attach boxed title to top left={yshift=-2mm,xshift=4mm},
  boxed title style={colback=KeyBorder,colframe=KeyBorder},
  left=4pt, right=4pt, top=4pt, bottom=4pt
]
\small
\begin{itemize}[leftmargin=1.2em,itemsep=2pt,topsep=1pt]
  \item \textbf{S3 convergence — inZORi FULL:} 99.1\% \;(CI95: [98.9\%, 99.3\%],\;30 seeds)
  \item \textbf{S3 convergence — Baseline A (warm-start):} 96.0\% \;(CI95: [95.8\%, 96.1\%])
  \item \textbf{S3 convergence — Baseline B (periodic reset):} 83.9\% \;(CI95: [83.7\%, 84.2\%])
  \item \textbf{N-1 topology recovery:} inZORi 1.82 steps vs Baseline B 12.91 steps ($\sim$7$\times$ faster)
  \item \textbf{Frozen-Elite-16 (runtime mode):} S3\,=\,100\% \;|\; 443.7 conv-steps/sec \;|\; avg\_k\_used\,=\,1.0
\end{itemize}
\end{tcolorbox}

\vspace{0.5em}

%% ── 1. Introduction ──────────────────────────────────────────────────────────
\section{Introduction}
\label{sec:intro}

Modern power grids face increasing variability from renewable generation,
demand fluctuation, and rare but high-impact contingency events such as
single-component outages (N-1 events).  Online power-flow (PF) solvers must
converge reliably within tight iteration budgets—typically $\leq$8 Newton-Raphson
iterations per state—to keep pace with SCADA refresh rates of 1–5 seconds per
measurement cycle \citep{Wood2013powergen,Zimmermann2011matpower}.

Classical approaches fall into two classes: \emph{warm-start tracking}, which
reuses the previous solution as the initial guess, and \emph{periodic-reset},
which periodically restarts from a flat initial point.  Both rely on fixed
heuristic parameters and fail systematically when the system undergoes rapid
topology changes or severe multi-zone shocks.

Evolutionary and neural approaches have been proposed for related problems,
including reinforcement-learning-based PF initialisation
\citep{Huang2021rlpf,Deka2019topo} and meta-learning warm-starts
\citep{Donon2020neuralPF}.  However, these require large offline datasets,
non-trivial training pipelines, and lack interpretable deployment artefacts.

\textbf{Our contribution} is different: we investigate whether a
\emph{pure evolutionary} framework—with no neural networks, no offline datasets,
no explicit PF domain knowledge injected beyond the residual feedback—can
self-organise effective NR parameter strategies in real-time streaming.
Our specific contributions are:

\begin{enumerate}[leftmargin=1.5em,itemsep=1pt]
  \item A seven-stage experimental study on a 118-bus nonlinear surrogate,
        with controlled multi-zone shocks, seasonal drift, and N-1 topology events.
  \item Two fairness controls (equal PF budget; equal wall-clock) ensuring that
        observed gains are not artefacts of compute advantage.
  \item A \textbf{Frozen-Elites} runtime mode: pre-trained, fixed genomes
        deployed without online evolution, outperforming both baselines while
        using half the PF compute cost.
  \item An ablation study showing that contextual elite selection
        and warm-start initialisation each contribute to performance,
        but Frozen-Elite quality alone suffices for 98.8\% S3 convergence.
  \item A phase-transition analysis revealing a sharp convergence cliff at
        $\text{nr\_max}$\,=\,5/6, clarifying the boundary of the compute-limited regime.
\end{enumerate}

\textbf{Scope}: This phase is restricted to the 118-bus streaming nonlinear
surrogate model only.  The surrogate uses controlled amplifications to induce
feasibility cliffs; it is inspired by PFΔ benchmark motivation \citep{Rivera2025pfidelta}
but does not reproduce full AC power flow physics.

%% ── 2. Related Work ──────────────────────────────────────────────────────────
\section{Related Work}
\label{sec:related}

\paragraph{Power flow solvers and initialisation.}
Newton-Raphson is the workhorse for PF in both offline and online settings
\citep{Wood2013powergen,Zimmermann2011matpower}.  Flat-start and warm-start
are standard heuristics \citep{Milano2010pf}.  Optimal initialisation
under streaming conditions remains an open problem.

\paragraph{Learning-based warm-starts.}
\citet{Donon2020neuralPF} use graph neural networks to predict PF solutions as
warm-starts; \citet{Huang2021rlpf} apply reinforcement learning for grid
control; \citet{Deka2019topo} use topology-aware regression.
These works require labelled datasets and assume relatively static network topology.

\paragraph{Evolutionary computation for power systems.}
Genetic algorithms and evolutionary strategies have been applied to optimal
power flow \citep{Todorovski2006ga,Vlachogiannis2009ga}, unit commitment
\citep{Saber2011ea}, and reactive power dispatch \citep{Liang2015ea}.
None of these works address the streaming, compute-limited NR initialisation
problem we study here.

\paragraph{Online adaptation without re-training.}
The idea of \emph{warm-started} or \emph{frozen} model deployment appears in
continual learning \citep{Kirkpatrick2017ewc} and meta-learning \citep{Finn2017maml}.
Our Frozen-Elites mode is conceptually closer to meta-learning
test-time adaptation, but uses evolved discrete phenotypes rather than
gradient-based meta-parameters.

\paragraph{The PFΔ benchmark motivation.}
This work is directly inspired by the PFΔ benchmark framework
\citep{Rivera2025pfidelta}, which proposes standardised streaming power-flow
scenarios for evaluating adaptive solvers.  Our surrogate is designed to
capture the compute-limited, multi-shock dynamics described in that work.

%% ── 3. Problem Formulation ───────────────────────────────────────────────────
\section{Problem Formulation}
\label{sec:problem}

\subsection{Streaming Power Flow Setting}
\label{ssec:streamingpf}

Let $G = (V, E)$ denote the 118-bus network graph with $|V| = 118$ nodes and
$|E|$ transmission lines.  At each discrete time step $t$, the system state is
described by a power injection vector $\mathbf{p}(t) \in \mathbb{R}^{|V|}$
and a voltage vector $\mathbf{v}(t) \in \mathbb{R}^{|V|}$.

The nonlinear power-balance residual is
\begin{equation}
  \mathbf{F}(\mathbf{v}) = \mathbf{p}_{\text{inj}}(\mathbf{v}) - \mathbf{p}(t) = 0\,,
\end{equation}
solved by Newton-Raphson iteration with Jacobian $J$:
\begin{equation}
  \mathbf{v}^{(k+1)} = \mathbf{v}^{(k)} - \alpha\, J^{-1}\mathbf{F}(\mathbf{v}^{(k)})\,,
\end{equation}
where $\alpha$ is the damping factor (step size).  Convergence is declared
when $\|\mathbf{F}(\mathbf{v})\|_\infty < \epsilon$ (tolerance $\epsilon = 10^{-4}$)
within $\text{nr\_max}$ iterations.

\subsection{Surrogate Nonlinearity and Stress Model}
\label{ssec:surrogate}

To create a genuine compute-limited regime (convergence is not trivially achievable
by any strategy), we augment the nominal PF residual with:
\begin{equation}
  \mathbf{F}_{\text{aug}}(\mathbf{v}) = \mathbf{F}(\mathbf{v})
    + \underbrace{\gamma\,\mathbf{v}^{\circ 3}}_{\text{cubic term}}
    + \underbrace{\delta_{\text{inj}}(t)}_{\text{mismatch injection}}\,,
\end{equation}
where $\gamma = 0.02$ and $\mathbf{v}^{\circ 3}$ denotes element-wise cube.
During severe shocks, we additionally weaken diagonal dominance of $J$ by a
factor $\beta_{\text{sev}} = 0.7$ and apply $\mathbf{v}$-decay on skipped PF
steps ($v_{\text{prev}} \leftarrow 0.98\,v_{\text{prev}}$).  These mechanisms
are \emph{controlled experimental amplifications} used to induce feasibility
cliffs under strict iteration budgets; they are inspired by PFΔ's motivation
but do not claim identical AC PF physics.

\subsection{Severity Regime}
\label{ssec:severity}

We adopt a four-scenario structure, where S3 (\emph{severe\_plus}) is the
primary evaluation target:
\begin{itemize}[itemsep=1pt]
  \item \textbf{S0} (mild): $\text{sev\_mult}=1.0$, seasonal drift only.
  \item \textbf{S1} (medium): $\text{sev\_mult}=3.5$, single-zone shocks.
  \item \textbf{S2} (severe): $\text{sev\_mult}=5.0$, multi-zone shocks.
  \item \textbf{S3} (severe\_plus): $\text{sev\_mult}=7.0$, multi-zone with
        ramp-in/plateau/ramp-out profiles, $v$-decay enabled, N-1 topology
        shocks (interval $\approx 900$ steps, duration 25 steps,
        overlap probability 0.4 for simultaneous multi-shocks).
\end{itemize}

\subsection{Evaluation Criteria}
\label{ssec:criteria}

For strategy $s$ and severity $\ell$:
\begin{itemize}[itemsep=1pt]
  \item $\text{CR}_{s,\ell}$: convergence rate (fraction of steps solved within
        $\text{nr\_max}$ iterations).
  \item $\text{RT}_{s}$: mean recovery time (steps from N-1 event end until
        stable convergence resumes).
  \item $\text{PBL}_{s}$: mean power balance limit residual at convergence.
  \item $\text{Throughput}_{s}$: converged S3 steps per wall-clock second.
\end{itemize}

Pass criteria for claiming a strategy is ``robust'':
\begin{equation*}
  \text{CR}_{s,\text{S3}} \geq 99\%
  \quad\text{and}\quad
  \text{CI}_{95}(s) \cap \text{CI}_{95}(\text{Baseline B}) = \emptyset\,,
\end{equation*}
persistent under both fairness controls.

%% ── 4. The inZORi Framework ──────────────────────────────────────────────────
\section{The inZORi Evolutionary Framework}
\label{sec:inzori}

\subsection{Overview}
\label{ssec:overview}

inZORi \citep{ZORFramework2025} is an autonomous multi-organism evolutionary
framework.  Each \emph{organism} maintains a \emph{genome}—a compact parameter
vector encoding solver strategy—and competes for influence over the system
through an energy-based selection mechanism.

\subsection{Genome and Phenotype}
\label{ssec:genome}

Each organism $i$ carries a genome
$\mathbf{g}_i = (\alpha_i, \mu_i, \rho_i, \text{jump}_i, \text{risk}_i)$, where:
\begin{itemize}[itemsep=1pt]
  \item $\alpha_i \in [0.3, 1.0]$: NR damping factor.
  \item $\mu_i \in [0.0, 1.0]$: warm-start mixture coefficient
        ($\mathbf{v}_0 = \mu_i\,\mathbf{v}_{\text{prev}} + (1-\mu_i)\,\mathbf{v}_{\text{flat}}$).
  \item $\rho_i \in [0.0, 1.0]$: memory learning rate (biological retention).
  \item $\text{jump}_i \in [0,1]$: probability of bold parameter mutation.
  \item $\text{risk}_i \in [0,1]$: risk tolerance affecting selection weight.
\end{itemize}

\subsection{World-Level Evaluation (Top-K)}
\label{ssec:topk}

At each step $t$, the \emph{world engine} scores all organisms by an energy
function combining recent convergence rate, mean PBL, and iteration economy.
The top-$K$ ($K=2$ online) organisms propose their genome-derived parameters;
the world solves PF once per selected organism and aggregates the feedback.
This world-level architecture reduced per-step compute by 40–50$\times$ compared
to per-organism PF evaluation, enabling large-scale validation.

\subsection{Energy-Based Selection and Biological Memory}
\label{ssec:energy}

Organism energy evolves as:
\begin{equation}
  E_i(t+1) = \rho_i\,E_i(t) + (1-\rho_i)\,r_i(t)\,,
\end{equation}
where $r_i(t) \in \{+1, 0, -1\}$ encodes the outcome of the PF solve at step $t$
(converged, skipped, diverged).  Organisms with sustained high energy reproduce
via crossover and mutation; low-energy organisms are pruned.

\subsection{Frozen-Elites Deployment Mode}
\label{ssec:frozen}

After an offline evolution run (50\,000 steps, 30 seeds), the top-16 organisms
by (energy + win-rate) are \emph{frozen}: evolution, mutation, and reproduction
are disabled.  At runtime:
\begin{equation*}
  i^*(t) = \arg\max_{i \in \mathcal{E}_{16}}\; \text{score}(i, c(t))\,,
\end{equation*}
where $c(t) = (\text{season}(t),\, N\text{-1 active}(t),\, \text{multi-shock active}(t),\,
\text{prev non-conv}(t))$ is a five-dimensional context vector and
$\text{score}(\cdot)$ is a deterministic linear rule exported as
\texttt{genome\_selection\_policy.json}.

The K-adaptive rule (K=2 only when any of: previous step non-convergent,
near iteration budget, N-1 + multi-shock overlap, recent non-convergence cooldown)
ensures cost-awareness: $\text{avg\_k\_used} < 1.25$ at deployment.

%% ── 5. Experimental Design ───────────────────────────────────────────────────
\section{Experimental Design}
\label{sec:design}

\subsection{Baselines}
\label{ssec:baselines}

We compare against two deterministic baselines with fixed heuristic parameters:
\begin{itemize}[itemsep=2pt]
  \item \textbf{Baseline A (warm-start tracking):} $\alpha=0.8$,
        $\mathbf{v}_0 = \mathbf{v}_{\text{prev}}$ at all times.
        Represents conventional warm-start NR without adaptation.
  \item \textbf{Baseline B (periodic-reset):} $\alpha=0.6$,
        $\mathbf{v}_0 = \mathbf{v}_{\text{flat}}$ every 500 steps
        (flat-start from nominal voltages).
        Represents a conservative restart strategy.
\end{itemize}

\subsection{Fairness Controls}
\label{ssec:fairness}

All strategies are compared under:
\begin{itemize}[itemsep=1pt]
  \item \textbf{Fairness A (equal PF budget):} baselines receive the same mean
        number of PF solver calls per step as inZORi.
  \item \textbf{Fairness B (equal wall-clock):} all strategies run for the same
        wall-clock duration; convergence-rate-per-second is reported.
\end{itemize}

\subsection{Configuration Summary}
\label{ssec:config}

\begin{table}[h]
\centering
\caption{Full-run experimental configuration.}
\label{tab:config}
\begin{tabular}{@{}ll@{}}
\toprule
Parameter & Value \\
\midrule
Network & IEEE 118-bus \\
Steps per run & 50\,000 \\
Seeds (FULL) & 30 \\
Seeds (Frozen) & 20 \\
\texttt{nr\_max} & 8 (primary), 12 (audit) \\
Tolerance $\epsilon$ & $10^{-4}$ \\
\texttt{severity\_multiplier} & 7.0 (severe\_plus) \\
\texttt{gamma} ($v^3$ nonlinearity) & 0.02 \\
$v$-decay rate (skipped steps) & 0.98 \\
N-1 interval & $\sim$900 steps \\
N-1 duration & 25 steps \\
Multi-shock overlap prob. & 0.40 \\
Fairness A PF budget (per step) & Equal to inZORi mean \\
Framework & inZORi v1.0 \\
\bottomrule
\end{tabular}
\end{table}

\subsection{Statistical Reporting}
\label{ssec:stats}

All metrics are reported as mean $\pm$ CI95 across seeds:
\begin{equation}
  \text{CI}_{95} = \bar{x} \pm 1.96\,\frac{\sigma}{\sqrt{n}}\,.
\end{equation}
We require non-overlapping CI95 intervals as a minimum condition for claiming
statistically significant differentiation.  Severe-event counts are validated:
$\geq$70 severe shocks per seed ($\geq$2\,100 total), $\geq$20 N-1 topology
events per seed (deterministic).

%% ── 6. Results ───────────────────────────────────────────────────────────────
\section{Results}
\label{sec:results}

\subsection{Stage 1–2: Calibration and Severe\_Plus Regime}
\label{ssec:s12}

In Stage 1 (mild stress, $\text{sev\_mult}=3.5$, 5k steps), all three strategies
converge at $\approx$100\%, confirming the regime is not yet discriminative.
Stage 2 introduces \texttt{severity\_multiplier}=7.0, cubic nonlinearity, and
$v$-decay, creating the compute-limited regime studied hereafter.

\subsection{Stage 3–4: Full Run (30 Seeds, 50k Steps)}
\label{ssec:s34}

Table~\ref{tab:main} reports the main results under Fairness A (equal PF budget,
$\text{nr\_max}=8$).  Figure~\ref{fig:main} provides visual comparisons of
S3 convergence, N-1 topology recovery, and power balance limit.

\begin{table}[ht]
\centering
\caption{Main results: 118-bus severe\_plus + N-1, 30 seeds, 50k steps, Fairness A.
All convergence rates are percentages; CI95 in brackets.}
\label{tab:main}
\begin{tabular}{@{}lcccccc@{}}
\toprule
Strategy & Global CR & S3 CR (\%) & S3 CI95 & Topo Rec. (steps) & Iter Mean & PBL Mean \\
\midrule
Baseline A & 98.4 & 96.0 & [95.8, 96.1] & 3.17 & 4.85 & $4.0\times10^{-6}$ \\
Baseline B & 93.6 & 83.9 & [83.7, 84.2] & 12.91 & 6.26 & $1.6\times10^{-5}$ \\
\textbf{inZORi} & \textbf{99.6} & \textbf{99.1} & \textbf{[98.9, 99.3]} & \textbf{1.82} & \textbf{4.54} & $\mathbf{1.1\times10^{-6}}$ \\
\bottomrule
\end{tabular}
\end{table}

\textbf{Key findings.}
\begin{itemize}[itemsep=2pt]
  \item inZORi achieves \textbf{+15.2\,pp} improvement in S3 convergence versus Baseline B
        (99.1\% vs 83.9\%) and \textbf{+3.1\,pp} versus Baseline A (99.1\% vs 96.0\%).
  \item CI95 intervals do not overlap between inZORi and Baseline B, confirming
        statistical significance.
  \item inZORi uses \emph{fewer} NR iterations on average (4.54 vs 4.85/6.26),
        demonstrating that convergence improvement is not obtained by consuming
        more compute per step.
  \item PBL is $3.75\times$ lower for inZORi vs Baseline A, indicating
        higher solution quality.
\end{itemize}

\textbf{Fairness B (equal wall-clock, 48.8\,s):}
inZORi achieves 0.0202 converged S3 steps/second vs 0.0197 (Baseline A) and
0.0172 (Baseline B), confirming that the improvement persists even when
normalised for runtime.

\textbf{Event counts.}
2\,215 severe shocks total (73.8 per run, CI95 [71.3, 76.4]);
660 topology events total (22.0 per run, deterministic).

\textbf{Audit at $\text{nr\_max}=12$:}
All strategies converge at $\approx$100\%, confirming that the differentiation
at $\text{nr\_max}=8$ reflects genuine compute-limited adaptation, not iteration
starvation of baselines.

\begin{figure}[ht]
\centering
\includegraphics[width=\linewidth]{figures/fig1_main_comparison.pdf}
\caption{Main comparison across three strategies (30 seeds, 50k steps, severe\_plus + N-1,
Fairness A): (a) S3 convergence rate with CI95 error bars; (b) mean N-1 topology
recovery time with CI95; (c) mean power balance limit (lower is better).
inZORi achieves +15.2\,pp vs Baseline B and recovers $\sim$7$\times$ faster after N-1 events.}
\label{fig:main}
\end{figure}

\subsection{Stage 5: Frozen-Elites Runtime}
\label{ssec:frozen}

Table~\ref{tab:frozen} reports all Frozen-Elite variants alongside baselines
(20 seeds, $\text{pf\_interval}=1$, Fairness A).
Figure~\ref{fig:frozen} provides visual comparison of S3 convergence and runtime throughput.

\begin{table}[ht]
\centering
\caption{Frozen-Elites runtime results: 118-bus severe\_plus + N-1, 20 seeds,
50k steps, pf\_interval=1, Fairness A.}
\label{tab:frozen}
\begin{tabular}{@{}lcccccc@{}}
\toprule
Strategy & S3 CR (\%) & S3 CI95 & Topo Rec. & Iter & Conv/Sec (S3) & Avg K \\
\midrule
Baseline A          & 96.0  & [95.8, 96.1] & 2.91  & 4.84 & 340.4 & 1.0 \\
Baseline B          & 86.0  & [85.7, 86.2] & 12.00 & 5.48 & 258.6 & 1.0 \\
inZORi online       & 99.7  & [99.5, 99.9] & 1.13  & 3.84 & 246.8 & 2.0 \\
Frozen-Top1-K1      & 100.0 & [100, 100]   & 1.00  & 3.25 & 434.4 & 1.0 \\
Frozen-Top8-K1      & 100.0 & [100, 100]   & 1.00  & 4.27 & 358.3 & 1.0 \\
\textbf{Frozen-Top16-K1}     & \textbf{100.0} & [100, 100] & \textbf{1.00} & \textbf{3.25} & \textbf{443.7} & 1.0 \\
Frozen-Top16-Kadapt & 100.0 & [100, 100]   & 1.00  & 3.25 & 440.5 & 1.0 \\
\bottomrule
\end{tabular}
\end{table}

\textbf{Key findings.}
\begin{itemize}[itemsep=2pt]
  \item All Frozen-Elite variants achieve 100\% S3 convergence with
        non-overlapping CI95 versus Baseline B ([100\%,\,100\%] vs [85.7\%,\,86.2\%]).
  \item Frozen-Top16-K1 achieves 443.7 converged S3 steps/second, exceeding
        both baselines (340.4 / 258.6) despite having zero online evolution overhead.
  \item Topology recovery is \emph{immediate} for all Frozen variants (1.00 step),
        versus 2.91 (Baseline A) and 12.00 (Baseline B).
  \item The K-adaptive rule yields avg\_k\_used\,=\,1.0 in this setting,
        confirming the deployment is cost-aware: no K=2 overheads were triggered.
  \item Frozenusing K=1 (one PF solve per step) vs inZORi online K=2 cuts
        PF compute cost in half while improving convergence from 99.7\% to 100\%.
\end{itemize}

\begin{figure}[ht]
\centering
\includegraphics[width=\linewidth]{figures/fig3_frozen_comparison.pdf}
\caption{Frozen-Elites vs online strategies: (a) S3 convergence with CI95 error bars;
(b) runtime throughput in converged S3 steps per second.
Frozen-Top16-K1 achieves the highest convergence and throughput simultaneously,
at K=1 (half the PF compute of inZORi online K=2).
Data source: 20 seeds, 50k steps, pf\_interval=1.}
\label{fig:frozen}
\end{figure}

\subsection{Stage 6: Ablation Study}
\label{ssec:ablation}

To verify that performance is not trivial, we test two ablations
(5k steps, 8 seeds) starting from Frozen-Elite-16 with contextual policy:

\begin{itemize}[itemsep=2pt]
  \item \textbf{Warm-start disabled:} S3 conv\,=\,100\%, iters\,=\,5.64.
        Frozen maintains 100\% even without warm-start advantage, but at
        $+73\%$ iteration cost (5.64 vs 3.25), confirming warm-start is
        helpful but not necessary.
  \item \textbf{Random elite selection (no policy):} S3 conv\,=\,98.8\%, iters\,=\,4.20.
        The 1.2\,pp drop confirms that contextual selection adds robustness;
        yet even random elite selection still outperforms Baseline B (98.8\% vs 86\%),
        confirming that elite genome quality, not just the selection rule,
        is the primary driver.
\end{itemize}

Figure~\ref{fig:ablation} summarises the ablation results.

\begin{figure}[ht]
\centering
\includegraphics[width=0.88\linewidth]{figures/fig4_ablation.pdf}
\caption{Ablation study: S3 convergence and NR iteration cost for Frozen-Elite-16
with full policy (left), warm-start disabled (centre), and random elite selection
(right). Results confirm that policy contributes +1.2\,pp and warm-start reduces
iteration cost by 42\%, but neither is strictly necessary for high convergence.}
\label{fig:ablation}
\end{figure}

\subsection{Stage 7: Phase-Transition Analysis}
\label{ssec:pareto}

Figure~\ref{fig:phase} shows S3 convergence as a function of $\text{nr\_max}$
for Frozen-Elite-16 and both baselines (5k steps, 8 seeds, three threshold variants).

\begin{table}[ht]
\centering
\caption{Phase transition: S3 convergence vs. NR budget for Frozen-Elite-16.}
\label{tab:phase}
\begin{tabular}{@{}cccl@{}}
\toprule
nr\_max & S3 CR (\%) & Iter Mean & Observation \\
\midrule
5 & 26.6 & 4.76 & Below cliff --- all strategies fail \\
6 & 100.0 & 4.34--4.46 & Above cliff --- evolved strategies succeed \\
7 & 100.0 & 4.34--4.45 & Stable plateau \\
8 & 100.0 & 3.25--4.85 & Production regime \\
\bottomrule
\end{tabular}
\end{table}

The sharp transition between $\text{nr\_max}=5$ (26.6\%) and $\text{nr\_max}=6$ (100\%)
confirms a genuine compute-limited regime: below the cliff, no strategy succeeds;
above the cliff, evolutionarily optimised strategies maintain 100\% while
Baseline B degrades to 84\% at $\text{nr\_max}=8$ (where the cliff is approached
for its fixed, non-adaptive heuristics).

\begin{figure}[ht]
\centering
\includegraphics[width=0.80\linewidth]{figures/fig2_phase_transition.pdf}
\caption{Phase transition in S3 convergence as a function of NR iteration budget.
Frozen-Top16-K1 (purple) crosses from failure ($<$27\%) to perfect convergence (100\%)
between nr\_max\,=\,5 and nr\_max\,=\,6.  Baselines degrade at nr\_max\,=\,8
where they approach the cliff from the non-adaptive direction.}
\label{fig:phase}
\end{figure}

\subsection{Pareto Frontier: Convergence vs. Iteration Cost}
\label{ssec:paretofig}

Figure~\ref{fig:pareto} plots each strategy in the
(mean NR iterations, S3 convergence) space,
with marker size proportional to throughput.
Frozen-Top16-K1 dominates all other strategies on both axes simultaneously.

\begin{figure}[ht]
\centering
\includegraphics[width=0.72\linewidth]{figures/fig6_pareto.pdf}
\caption{Pareto frontier: S3 convergence vs.\ mean NR iterations per solve.
Marker size is proportional to throughput (converged steps/sec).
Frozen-Top16-K1 ($\star$) Pareto-dominates all baselines and inZORi online.
The grey markers (nr\_max=5 below cliff, nr\_max=6 above) illustrate the
transition region.}
\label{fig:pareto}
\end{figure}

%% ── 7. Discussion ────────────────────────────────────────────────────────────
\section{Discussion}
\label{sec:discussion}

\subsection{Emergent Robustness from Evolutionary Pressure}
\label{ssec:emergence}

The inZORi framework discovers effective NR parameter combinations through
sustained interaction with the streaming environment, without any domain-specific
guidance beyond the binary convergence signal.  The +15.2\,pp improvement over
Baseline B demonstrates that evolutionary pressure toward sustained convergence
produces qualitatively different parameter strategies than fixed heuristics.

\subsection{The Value of Frozen Deployment}
\label{ssec:frozenvalue}

The Frozen-Elites result challenges a common assumption: that online adaptation
is necessary for robust streaming performance.  Once a high-quality elite pool
has been evolved offline, a simple deterministic selection rule (no ML, no RL)
suffices for 100\% S3 convergence.  The key insight is that:
\begin{enumerate}[itemsep=1pt]
  \item The elite pool encodes \emph{behavioural diversity} across operating regimes
        (mild, seasonal, severe, N-1).
  \item The contextual policy routes each step to the genome best suited for the
        current context, with O(1) compute overhead.
  \item K=1 PF solve per step is sufficient, halving compute vs.\ inZORi online.
\end{enumerate}
This has direct practical implications: a utility operator can run an overnight
evolutionary training run, extract the elite pool, and deploy a lightweight
runtime requiring no further training infrastructure.

\subsection{Practical Deployment Scenario}
\label{ssec:practical}

SCADA-connected grid operators refresh state estimates every 1–5 seconds.
At 443 converged steps/second (Frozen-Top16-K1), the runtime comfortably
exceeds any realistic SCADA refresh requirement, including under N-1 contingency
events where recovery is immediate (1 step $\approx$ milliseconds).

\subsection{Limitations}
\label{ssec:limits}

\begin{enumerate}[itemsep=2pt]
  \item \textbf{Surrogate model:} The cubic nonlinearity surrogate captures the
        difficulty structure of compute-limited PF but does not model full AC
        power flow physics (transformer models, generator reactive limits,
        voltage magnitude constraints).  Results demonstrate proof of concept
        on a tractable surrogate, not a production-ready grid solver.
  \item \textbf{Network topology:} Experiments use the IEEE 118-bus test case.
        Different topologies (radial distribution networks, meshed transmission)
        may require re-evolution of the elite pool.
  \item \textbf{Fixed N-1 patterns:} N-1 events are sampled from a fixed
        statistical model.  Cascading failures or correlated outages are not modelled.
  \item \textbf{No NN/RL comparison:} Deep RL and GNN-based warm-starts
        are out of scope; a direct comparison is deferred to future work.
\end{enumerate}

%% ── 8. Reproducibility ───────────────────────────────────────────────────────
\section{Reproducibility}
\label{sec:repro}

\begin{table}[ht]
\centering
\caption{Reproducibility run matrix.  All scripts located in
\texttt{problems/zor\_pfdelta\_stream\_118/}.}
\label{tab:repro}
\begin{tabular}{@{}lllccc@{}}
\toprule
Stage & Script & Mode & Seeds & Steps & Output Folder \\
\midrule
2--4 & \texttt{run\_conf\_118\_n1\_full.py} & FULL & 30 & 50\,000 & \texttt{conf\_118\_n1\_full\_once} \\
3 (gate) & \texttt{run\_conf\_118\_n1\_dev.py} & DEV & 8--12 & 5\,000 & \texttt{conf\_118\_n1\_dev} \\
5 & \texttt{run\_frozen\_elites\_runtime.py} & RUNTIME & 20 & 50\,000 & \texttt{conf\_frozen\_elites\_runtime} \\
6--7 & \texttt{run\_micro\_ablation.py} & MICRO & 8 & 5\,000 & \texttt{conf\_micro\_ablation} \\
\bottomrule
\end{tabular}
\end{table}

Results are aggregated by \texttt{results\_aggregator.py}.
All seeds are set via \texttt{numpy.random.seed(s)} for $s \in \{0,\ldots,\text{N}-1\}$.
All elite genomes and the selection policy are exported as JSON:
\texttt{elite\_genomes\_top16.json}, \texttt{genome\_selection\_policy.json}.
The full results are published at:
\url{https://dumitrunovic-svg.github.io/inZORi/tests/pfdelta_phase1_118/}.

%% ── 9. Conclusion ────────────────────────────────────────────────────────────
\section{Conclusion}
\label{sec:conclusion}

We have presented a seven-stage experimental study of the inZORi evolutionary
framework applied to streaming power-flow parameter adaptation under severe
multi-zone shocks and N-1 topology events.

On the 118-bus nonlinear surrogate with $\text{nr\_max}=8$ and a severity
multiplier of 7.0 (30 seeds, 50\,000 steps per run), inZORi achieves:
\begin{itemize}[itemsep=1pt]
  \item \textbf{99.1\% S3 convergence} (CI95 non-overlapping vs Baseline B),
        representing a +15.2\,pp improvement over periodic-reset heuristics.
  \item \textbf{$\sim$7$\times$ faster topology recovery} after N-1 events
        (1.82 vs 12.91 steps).
  \item \textbf{Equal or better compute efficiency} under both equal-budget
        and equal-time fairness controls.
\end{itemize}

The Frozen-Elites mode further demonstrates that a pre-trained elite pool,
deployed without any online evolution, achieves \emph{100\% S3 convergence}
at 443 converged steps/second—Pareto-dominating all online and baseline strategies
in both convergence and throughput.

\textbf{Phase 1 demonstrates that a pure evolutionary system can maintain
near-perfect feasibility under strict NR iteration budgets in a streaming
nonlinear regime with regional shocks and N-1 topology events, and can be
deployed in frozen form for real-time operation.}

These results motivate further investigation on full AC power flow formulations,
larger networks, and direct comparison with neural-network-based warm-start methods.

%% ── Acknowledgements ─────────────────────────────────────────────────────────
\section*{Acknowledgements}

This research was inspired by and directly builds upon the PFΔ benchmark
motivation introduced by \citet{Rivera2025pfidelta}.  The 118-bus test network
is from the standard MATPOWER distribution \citep{Zimmermann2011matpower}.
The author thanks the open-source scientific Python ecosystem (NumPy, SciPy,
Matplotlib) for the tools enabling this research.

%% ── References ───────────────────────────────────────────────────────────────
\bibliographystyle{plainnat}
\bibliography{references}

%% ── Appendix ─────────────────────────────────────────────────────────────────
\appendix

\section{Genome Parameter Ranges}
\label{app:genome}

\begin{table}[H]
\centering
\caption{Genome parameter search space for inZORi organisms.}
\begin{tabular}{@{}llll@{}}
\toprule
Parameter & Symbol & Range & Role \\
\midrule
Damping factor & $\alpha$ & [0.30, 1.00] & NR step size \\
Warm-start mix & $\mu$ & [0.00, 1.00] & Blend prev/flat init \\
Memory rate & $\rho$ & [0.00, 1.00] & Energy retention \\
Jump probability & $p_\text{jump}$ & [0.00, 1.00] & Bold mutation rate \\
Risk tolerance & $r$ & [0.00, 1.00] & Selection weight \\
\bottomrule
\end{tabular}
\end{table}

\section{Shock Generation Protocol}
\label{app:shocks}

Severe multi-zone shocks follow a trapezoidal time profile:
\begin{equation*}
  s(t) = \text{sev\_mult} \times
  \begin{cases}
    (t - t_\text{start})/t_\text{ramp} & t \in [t_\text{start}, t_\text{start}+t_\text{ramp}) \\
    1.0 & t \in [t_\text{start}+t_\text{ramp},\, t_\text{end}-t_\text{ramp}) \\
    (t_\text{end} - t)/t_\text{ramp} & t \in [t_\text{end}-t_\text{ramp},\, t_\text{end})
  \end{cases}
\end{equation*}
with $t_\text{ramp} = 30$ steps and plateau duration $\sim$250 steps.
Shocks affect 2–4 zones simultaneously with independent amplitude draws
from $\mathcal{U}(0.8, 1.0)$.

\section{K-Adaptive Trigger Logic}
\label{app:kadapt}

K-adaptive selects K=2 (two PF solves per step) when any of the following holds:
\begin{enumerate}[itemsep=1pt]
  \item Previous step was non-convergent.
  \item Current NR solve is within 1 iteration of budget limit.
  \item Active N-1 event overlaps with active multi-shock.
  \item A non-convergence event occurred within the last 5 steps (cooldown).
\end{enumerate}
Otherwise K=1.  In the 50k-step runtime evaluation (20 seeds),
none of these conditions triggered: avg\_k\_used\,=\,1.0.

\end{document}
